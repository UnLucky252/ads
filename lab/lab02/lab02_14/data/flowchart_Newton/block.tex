\documentclass{standalone}
\usepackage{tikz}  
\usepackage[russian]{babel}
\usetikzlibrary{graphs}
\usetikzlibrary{shapes, arrows}
\begin{document}

\tikzstyle{startstop} = [rectangle, rounded corners, minimum width=3cm, minimum height=1cm,text centered, draw=black, fill=red!30]
\tikzstyle{io} = [trapezium, trapezium left angle=70, trapezium right angle=110, minimum width=3cm, minimum height=1cm, text centered, draw=black, fill=blue!30]
\tikzstyle{decision} = [diamond, minimum width=3cm, minimum height=1cm, text centered, draw=black, fill=green!30]
\tikzstyle{arrow} = [thick,->,>=stealth]

\tikzstyle{process} = [rectangle, minimum width=4cm, minimum height=1cm, text centered, text width=5.2cm, draw=black, fill=orange!30]
\tikzstyle{recurcy} = [rectangle, minimum width=3cm, minimum height=1cm, text centered, text width=3cm, draw=black, fill=purple!30]

\begin{tikzpicture}[node distance=2cm]
\node (start) [startstop] {Start};
\node(pro0) [process, below of=start, yshift=-0.6cm] {F(X) = 0 \\ float eps \\n = 0 };
\draw [arrow] (start) -- (pro0);


\node(pro1) [process, below of=pro0, yshift=-0.6cm] {Находим a и b, такие что: F(a)*F(b) < 0, тогда наша искомая точка будет находиться в промежутке [a;b]
};
\draw [arrow] (pro0) -- (pro1);

\node(pro2) [process, below of=pro1, yshift=-0.6cm] {Xo = (a + b) / 2};
\draw [arrow] (pro1) -- (pro2);

\node(pro3) [process, below of=pro2, yshift=-0.6cm] {Xn+1 = Xn - (1/F'(X0))*F(Xn)\\ \vspace{0.3cm} n++};
\draw [arrow] (pro2) -- (pro3);

\node (dec0) [decision, below of=pro3, yshift=-3cm] {while (|Xn+1 - Xn| < eps)};

\draw [arrow] (pro3) -- (dec0);
\draw [arrow] (-4,-9) -- (0,-9);
\draw (-4,-9) -- (-4,-15.4);
\draw (-4,-15.4) -- node[anchor=south] {true} (-2.46, -15.4);

\draw (2.46,-15.4) -- node[anchor=south] {false} (4, -15.4);
\draw (4, -15.4) -- (4, -19.9);

\node (out1) [io, below of=dec0, yshift=-2.5cm] {X};
\draw [arrow] (4, -19.9) -- (1.2, -19.9);

\node (stop) [startstop, below of=out1, yshift=-0.5cm] {Stop};
\draw [arrow] (out1) -- (stop);

\end{tikzpicture}
\end{document}